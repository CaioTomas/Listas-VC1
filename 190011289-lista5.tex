\documentclass[12pt,a4paper]{article}
\usepackage{amsmath,amssymb,amsthm}
\usepackage{makeidx,graphics}
\usepackage[dvips]{graphicx}
%\usepackage[latin1]{inputenc}
%\usepackage[portuguese]{babel}
\usepackage[utf8]{inputenc}
\usepackage{ae}
\usepackage{indentfirst}
\usepackage{amsbsy}
\usepackage{fancyhdr}
\usepackage{pstricks}
\usepackage[all]{xy}
\usepackage{wrapfig}
\usepackage[pdfstartview=FitH,backref,colorlinks,bookmarksnumbered,bookmarksopen,linktocpage,urlcolor=blue,
linkcolor=cyan]{hyperref}
\usepackage{bussproofs}
\usepackage{amsmath}
\usepackage{mathtools}
\usepackage{amsthm}
\usepackage{amsfonts}
\usepackage{amssymb}
\usepackage{wasysym}
\usepackage{amsbsy}
\usepackage{url}
%\usepackage{subfigure}
\usepackage{subcaption}
\usepackage{pgfplots}
\pgfplotsset{compat=newest}
\usepgfplotslibrary{fillbetween}

\usepackage{esint}

\newtheorem{definition}{Definição}
%\newtheorem{example}{Exemplo}
\newtheorem{lema}{Lema}
\newtheorem{teorema}{Teorema}
\newtheorem{corolario}{Corolário}
\newtheorem*{obs}{Observação}

\setlength{\topmargin}{-1.0in}
\setlength{\oddsidemargin}{0in}
\setlength{\evensidemargin}{0in}
\setlength{\textheight}{10.5in}
\setlength{\textwidth}{6.5in}
\setlength{\baselineskip}{12mm}

\newcommand{\dx}{\ \mathrm{d} x }
\newcommand{\dy}{\ \mathrm{d} y }
\newcommand{\dz}{\ \mathrm{d} z }
\newcommand{\du}{\ \mathrm{d} u }
\newcommand{\dv}{\ \mathrm{d} v }
\newcommand{\dr}{\ \mathrm{d} r }
\newcommand{\dt}{\ \mathrm{d} t }
\newcommand{\dteta}{\ \mathrm{d} \theta }
\newcommand{\dro}{\ \mathrm{d} \rho }
\newcommand{\dfi}{\ \mathrm{d} \phi }
\newcommand{\ds}{\ \mathrm{d} s }
\newcommand{\dS}{\ \mathrm{d} S }
\newcommand{\dq}{\ \mathrm{d} q }
\newcommand{\dif}{\mathrm{d}}

\DeclareMathOperator{\rot}{rot}
\DeclareMathOperator{\diverg}{div}

\graphicspath{{img/}}

\renewcommand{\sectionmark}[1]{ \markright{ \thesection.\ #1}}

\title{\textbf{Variável Complexa 1}\\ Lista 5}
\author{Caio Tomás de Paula}
\date{\today}

\begin{document}
	\maketitle
	\begin{enumerate}
		\item Tomando 
		$$
		U = \left\{ z\in\mathbb{C} : \Im(z)<0 \right\},
		$$
		temos que $f$ está bem definida, pois as funções $\log$ e raiz quadrada (ramo principal) estarão bem definidas. 
		
		Agora, note que
		\begin{align*}
		\cos(f(z)) &= \frac{1}{2}\left( \exp\left( \log\left( z+\sqrt{z^2-1} \right) \right) + \exp\left( -\log\left( z+\sqrt{z^2-1} \right) \right) \right) \\
		&= \frac{1}{2}\left( z+\sqrt{z^2-1} + \frac{1}{z+\sqrt{z^2-1}} \right) \\
		&= \frac{1}{2}\left( z+\sqrt{z^2-1} + z-\sqrt{z^2-1} \right) \\
		&= z.
		\end{align*}
		O mesmo $U$ de $f$ vale para $g$, pelas mesmas considerações.
		
		Por fim, perceba que
		\begin{align*}
		\cos(g(z)) &= \frac{1}{2}\left( \exp\left( \log\left( z-\sqrt{z^2-1} \right) \right) + \exp\left( -\log\left( z-\sqrt{z^2-1} \right) \right) \right) \\
		&= \frac{1}{2}\left( z-\sqrt{z^2-1} + \frac{1}{z-\sqrt{z^2-1}} \right) \\
		&= \frac{1}{2}\left( z-\sqrt{z^2-1} + z+\sqrt{z^2-1} \right) \\
		&= z.
		\end{align*}
		
		\item Primeiro, note que 
		$$
		\sin(f(z)) = z \Longleftrightarrow \cos^2(f(z)) = 1 - z^2.
		$$
		Agora, da Regra da Cadeia segue que
		$$
		1 = \cos(f(z))\cdot f'(z) \Longleftrightarrow 1 = \cos^2(f(z))\cdot(f'(z))^2 \Longleftrightarrow (f'(z))^2 = \frac{1}{1-z^2}.
		$$
		Por fim, suponha que $\displaystyle{\frac{\pi}{2}\in f(U)}$. Então existe $z_1\in U$ tal que $f(z_1) = \pi/2$, de modo que
		$$
		\sin(f(z_1)) = 1 \Longrightarrow z_1 = 1.
		$$
		Ora, mas então como
		$$
		(f'(z))^2 = \frac{1}{1-z^2},
		$$
		teríamos que $z=1$ seria uma singularidade de $f$, contradizendo o fato de $f$ ser holomorfa em $U$. Portanto, $\pi/2\notin U$. 
		
		\item Temos
		\begin{align*}
		az^2+bz+c=0 \Leftrightarrow \left( z+\frac{b}{2a} \right)^2 = \frac{b^2-4ac}{4a^2}
		\end{align*}
		e, fazendo
		$$
		z_{\star} = z+\frac{b}{2a}, \ \ \ w = \frac{b^2-4ac}{4a^2},
		$$
		segue que os únicos valores possíveis de $z_{\star}$ são
		\begin{align*}
		z_{\star} &= |w|^{1/2}\cdot e^{i\arg(w)/2} \\
		z_{\star} &= |w|^{1/2}\cdot e^{i(\pi-\arg(w)/2)},
		\end{align*}
		de modo que os únicos valores possíveis de $z$ são
		\begin{align*}
		z &= -\frac{b}{2a} + |w|^{1/2}\cdot e^{i\arg(w)/2} \\
		z &= -\frac{b}{2a} + |w|^{1/2}\cdot e^{i(\pi-\arg(w)/2)},
		\end{align*}
		
		\item Note que
		$$
		z = 3^{\frac{1}{5}} \Longleftrightarrow z^5 = 3.
		$$
		Agora, perceba que podemos escrever
		\begin{align*}
		z^5 &= 3 = 3\cdot e^{i0} \Longrightarrow z = \sqrt[5]{3}\cdot e^{i0} = \sqrt[5]{3} \\
		z^5 &= 3 = 3\cdot e^{i2\pi} \Longrightarrow z = \sqrt[5]{3}\cdot e^{i2\pi/5} \\
		z^5 &= 3 = 3\cdot e^{i4\pi} \Longrightarrow z = \sqrt[5]{3}\cdot e^{i4\pi/5} \\
		z^5 &= 3 = 3\cdot e^{i6\pi} \Longrightarrow z = \sqrt[5]{3}\cdot e^{i6\pi/5} \\
		z^5 &= 3 = 3\cdot e^{i8\pi} \Longrightarrow z = \sqrt[5]{3}\cdot e^{i8\pi/5}
		\end{align*}
		qualquer que seja o ramo $\phi\neq \pi$ do logaritmo. Por fim, note que para se fizéssemos
		\begin{align*}
		z^5 &= 3 = 3\cdot e^{i10\pi} \Longrightarrow z = \sqrt[5]{3}\cdot e^{i10\pi/5} = \sqrt[5]{3}
		\end{align*} 
		e assim por diante, continuando a considerar os múltiplos de $2\pi$ acima de $8\pi$, passaremos a repetir as $5$ raízes acima. Portanto, concluímos que $3^{\frac{1}{5}}$ só pode assumir $5$ valores distintos.
		
		\item Usando o mesmo argumento acima, note que
		$$
		z = 3^{\sqrt{2}} \Longleftrightarrow z^{\sqrt{2}} = 9.
		$$
		Além disso, podemos escrever
		\begin{align*}
		z^{\sqrt{2}} = 9\cdot e^{i0} &\Longrightarrow z = 3^{\sqrt{2}}\cdot e^{i0} \\
		z^{\sqrt{2}} = 9\cdot e^{i2\pi} &\Longrightarrow z = 3^{\sqrt{2}}\cdot e^{i2\pi/\sqrt{2}} \\
		z^{\sqrt{2}} = 9\cdot e^{i4\pi} &\Longrightarrow z = 3^{\sqrt{2}}\cdot e^{i4\pi/\sqrt{2}} \\ 
		\end{align*}
		e assim por diante, para qualquer múltiplo inteiro de $2\pi$ e qualquer que seja o ramo $\phi\neq \pi$ do logaritmo. Contudo, note que nunca haverá uma repetição, já que $\nexists j\in\mathbb{Z}$ tal que
		$$
		\frac{2k\pi}{\sqrt{2}} = 2j\pi.
		$$
		Assim, há infinitos valores possíveis para $3^{\sqrt{2}}$.
		
		\item[12.] A primeira desigualdade claramente não vale para todo $z\in\mathbb{C}$, pois tomando $z=2$, temos
		$$
		|e^2| > 4 > 2,
		$$
		de modo que não vale 
		$$
		|e^z|\leq |z|, \ \forall z\in\mathbb{C}.
		$$
		Por outro lado, tomando $z=ni, n\in\mathbb{N}$, temos 
		$$
		|z| = n \geq 1 = |e^z|,
		$$
		de modo que também não vale 
		$$
		|e^z|\geq |z|, \ \forall z\in\mathbb{C}.
		$$
		Agora, tomando $z=a+ib, a,b\in\mathbb{R}$, temos que
		$$
		e^{|z|} = e^{\sqrt{a^2+b^2}}
		$$
		e
		$$
		|z| = \sqrt{a^2+b^2}\in\mathbb{R}.
		$$
		Ora, mas $e^x>x, \ \forall x\in\mathbb{R}$. Portanto, segue que
		$$
		e^{|z|} \leq |z|
		$$
		é falsa. Por fim, tomando novamente $z=a+ib, a,b\in\mathbb{R}$. Temos
		$$
		|e^z| = |e^a\cdot e^{ib}| = e^a
		$$
		e
		$$
		e^{|z|} = e^{\sqrt{a^2+b^2}}\geq e^{\sqrt{a^2}} = e^{|a|}\geq e^a\text{, já que } |a|\geq a, \ \forall a\in\mathbb{R}.
		$$
		Portanto, vale a última desigualdade:
		$$
		e^{|z|}\geq |e^z|, \ \forall  z\in\mathbb{C}
		$$
		
		\item [14.] Note que
		$$
		0 < \frac{n!}{n^n} = \underbrace{\frac{n}{n}}_{=1}\underbrace{\cdot\frac{n-1}{n}\cdots\frac{2}{n}}_{<1}\cdot\frac{1}{n} < \frac{1}{n}.
		$$
		Como $\displaystyle{\lim\limits_{n\to\infty} 0 = 0 = \lim\limits_{n\to\infty}\frac{1}{n}}$, segue do Teorema do Sanduíche que $\displaystyle{\lim\limits_{n\to\infty} \frac{n!}{n^n} = 0}$. 
		
		\item[18.]
		\begin{enumerate}
			\item[(c)] Temos
			$$
			\frac{\log(ni)}{ni} = \frac{\ln|ni| + i\arg(ni)}{ni} = -i\frac{\ln(n)}{n} + \frac{\pi}{2n}.
			$$
			Como
			\begin{align*}
			\lim\limits_{n\to\infty} \frac{\pi}{2n} = 0 
			\end{align*}
			e
			\begin{align*}
			\lim\limits_{n\to\infty} \frac{\ln(n)}{n} \stackrel{\text{L'H}}{=} \lim\limits_{n\to\infty} \frac{1}{n} = 0
			\end{align*}
			segue que
			\begin{align*}
			\lim\limits_{n\to\infty} \frac{\log(ni)}{ni} = 0. 
			\end{align*}
			
			\item[(e)] Note que
			$$
			\lim\limits_{n\to\infty}\sqrt[ni]{ni} = \lim\limits_{n\to\infty} \sqrt[ni]{\exp\left( \log(ni) \right)} = \lim\limits_{n\to\infty}\exp\left( \frac{\log(ni)}{ni} \right) = \exp\left( \lim\limits_{n\to\infty}\frac{\log(ni)}{ni} \right) = e^0 = 1.
			$$
		\end{enumerate}
		
		\item[23.] Para a primeira série, note que
		$$
		\sum_{n=1}^{\infty}\frac{1}{ni} = -i\sum_{n=1}^{\infty}\frac{1}{n},
		$$
		ou seja, a série desejada é um múltiplo escalar da série harmônica. Portanto, como a série harmônica diverge, a série procurada diverge.
		
		Agora, observe que
		$$
		\sum_{n=1}^{\infty}\frac{1}{n+i} = \sum_{n=1}^{\infty}\left(\frac{n}{n^2+1} - i\frac{1}{n^2+1}\right) 
		$$
		converge se, e só se, as séries
		$$
		\sum_{n=1}^{\infty}\frac{n}{n^2+1} \ \text{ e } \ \sum_{n=1}^{\infty}-\frac{1}{n^2+1}
		$$
		convergem. Ora, mas aplicando o teste da integral à primeira série, temos que ela converge se, e só se
		$$
		\int_{1}^{\infty}\frac{x}{1+x^2}dx < \infty .
		$$
		Calculando a integral, temos
		$$
		\int_{1}^{\infty}\frac{x}{1+x^2}dx = \frac{1}{2}\cdot\int_{1}^{\infty}\frac{1}{1+u}du = \frac{1}{2}\cdot\ln(1+u)\Big|_{1}^{\infty} = +\infty,
		$$
		de modo que a série 
		$$
		\sum_{n=1}^{\infty}\frac{n}{n^2+1}
		$$
		diverge e, por consequência, a série
		$$
		\sum_{n=1}^{\infty}\frac{1}{n+i}
		$$
		também diverge.
		
		\item[24.]
		\begin{enumerate}
			\item Temos
			$$
			\left| \frac{a_n}{a_{n+1}} \right| = \left| \frac{n}{n+1}\cdot 3i \right| = \frac{3n}{n+1} \xrightarrow{n\to\infty} 3
			$$
			de modo que o raio de convergência de 
			$$
			\sum_{n=0}^{\infty}\frac{n}{(3i)^n}(z-1)^n = \sum_{n=1}^{\infty}\frac{n}{(3i)^n}(z-1)^n
			$$
			é $3$, e a série converge para todo $z\in D(1,3)$.
			
			\item Temos
			$$
			\left| \frac{a_n}{a_{n+1}} \right| = \left| \frac{11^{n+2i}}{n!}\cdot \frac{(n+1)!}{11^{n+1+2i}} \right| = \left| \frac{n+1}{11} \right| = \frac{n+1}{11} \xrightarrow{n\to\infty} \infty 
			$$
			de modo que o raio de convergência de 
			$$
			\sum_{n=0}^{\infty}\frac{11^{n+2i}}{n!}z^n 
			$$
			é $\infty$, e a série converge para todo $z\in \mathbb{C}$.
			
			\item Note que 
			\begin{align*}
			\left| \frac{n^{2i}}{2^n} \right|^{1/n} &= \frac{\left| n^{2i} \right|^{1/n}}{2} \\ 
			&= \frac{\left| \exp\left( 2i\ln(n) \right) \right|^{1/n}}{2} \\
			&= \frac{\left| ne^{2i} \right|^{1/n}}{2} \\
			&= \frac{\sqrt[n]{n}}{2} \\
			&= \frac{1}{2}\exp\left(\frac{\ln(n)}{n}\right) \xrightarrow{n\to\infty}\frac{1}{2}e^0 = \frac{1}{2}
			\end{align*} 
			de modo que o raio de convergência $R$ é
			$$
			R = \frac{1}{\lim\limits_{n\to\infty}\sqrt[n]{|a_n|}} = \frac{1}{\frac{1}{2}} = 2,
			$$
			e a série converge para todo $z\in D(\pi,2)$.
			
			\item Temos que o raio de convergência $R$ é
			\begin{align*}
			R = \lim\limits_{n\to\infty} \left| \frac{5}{(4+3i)^n}\cdot\frac{(4+3i)^{n+1}}{5} \right| = \lim\limits_{n\to\infty} |4+3i| = 5,
			\end{align*}
			e a série converge para todo $z\in D(0,5)$.
			
			\item Temos que o raio de convergência $R$ é
			\begin{align*}
			R = \lim\limits_{n\to\infty} \left| \frac{7n}{(5+i)^n}\cdot\frac{(5+i)^{n+1}}{7(n+1)} \right| = \lim\limits_{n\to\infty} |5+i|\cdot\frac{7n}{7n+1} = \sqrt{26},
			\end{align*}
			e a série converge para todo $z\in D(-2,\sqrt{26})$.
			
			\item Temos que o raio de convergência $R$ é
			\begin{align*}
			R &= \lim\limits_{n\to\infty} \left| \frac{\log((n+1)i)}{\log(ni)} \right| \\
			&= \lim\limits_{n\to\infty}\sqrt{\frac{\ln^2(n+1) + \pi^2/4}{\ln^2(n) + \pi^2/4}} \\
			&= \sqrt{\lim\limits_{n\to\infty} \frac{2\ln(n+1)\frac{1}{n+1}}{2\ln(n)\frac{1}{n}} } \\
			&= \lim\limits_{n\to\infty} \sqrt{\frac{n}{n+1}}\cdot\sqrt{\lim\limits_{n\to\infty}\frac{\ln(n+1)}{\ln(n)}} \\
			&= \sqrt{\lim\limits_{n\to\infty}\frac{n}{n+1}} = 1,
			\end{align*}
			e a série converge para todo $z\in D(0,1)$.
			
			\item Temos que o raio $R$ de convergência é
			$$
			R = \lim\limits_{n\to\infty}\left| \frac{i^n}{2^{ni}}\cdot\frac{2^{(n+1)i}}{i^{n+1}} \right| = \lim\limits_{n\to\infty}\left| \frac{2^i}{i} \right| = |2^i| = |\exp(i\ln2)| = 1,
			$$
			e a série converge para todo $z\in D(0,1)$.
			
			\item Temos que o raio $R$ de convergência é
			$$
			R = \lim\limits_{n\to\infty} \left| \frac{(-1)^{n+1}}{n\sqrt{3i}}\cdot\frac{(n+1)\sqrt{3i}}{(-1)^{n+2}} \right| = \lim\limits_{n\to\infty}\frac{n+1}{n} = 1,
			$$
			e a série converge para todo $z\in D(0,1)$.
			
			\item Temos que o raio $R$ de convergência é
			$$
			R = \lim\limits_{n\to\infty}\left| \frac{3^{ni}}{i(2n)!}\cdot\frac{i(2n+2)!}{3\cdot 3^{ni}} \right| = \lim\limits_{n\to\infty} \frac{(2n+2)(2n+1)}{|3^i|} = +\infty, 
			$$
			ou seja, a série converge para todo $z\in\mathbb{C}$.
			
			\item Temos que o raio $R$ de convergência é 
			$$
			R = \lim\limits_{n\to\infty}\left| \frac{(n!)^2}{(2n)!}\cdot\frac{(2n+2)!}{[(n+1)!]^2} \right| = \lim\limits_{n\to\infty}\frac{(2n+2)(2n+1)}{(n+1)^2} = \lim\limits_{n\to\infty} \frac{4n^2+6n+2}{n^2+2n+1} = 4,
			$$
			e a série converge para todo $z\in D(0,4)$.
			
			\item Temos que o raio $R$ de convergência é
			$$
			R = \frac{1}{\lim\limits_{n\to\infty} \left| \left( 1+\frac{1}{n} \right)^n \right|^{1/n} } = \frac{1}{\lim\limits_{n\to\infty} 1 + \frac{1}{n}} = 1,
			$$
			e a série converge para todo $z\in D(0,1)$.
			
			\item Temos que o raio $R$ de convergência é
			$$
			R = \frac{1}{\lim\limits_{n\to\infty} \left| \left( 1+\frac{1}{n} \right)^{n^2} \right|^{1/n} } = \frac{1}{\lim\limits_{n\to\infty} \left(1 + \frac{1}{n}\right)^n} = \frac{1}{e},
			$$
			e a série converge para todo $z\in D(0,1/e)$.
			
			\item Temos que o raio $R$ de convergência é
			$$
			R = \frac{1}{\lim\limits_{n\to\infty} \left| \left( 1-\frac{1}{n} \right)^n \right|^{1/n} } = \frac{1}{\lim\limits_{n\to\infty} 1 - \frac{1}{n}} = 1,
			$$
			e a série converge para todo $z\in D(0,1)$.
			
			\item Temos que o raio $R$ de convergência é
			$$
			R = \frac{1}{\lim\limits_{n\to\infty}\left| n^{\ln(n)} \right|^{1/n}} = \frac{1}{\lim\limits_{n\to\infty} e^{\frac{\ln^2(n)}{n}}} = \frac{1}{\exp\left( \lim\limits_{n\to\infty} \frac{\ln^2(n)}{n} \right)} = \frac{1}{\exp\left(\lim\limits_{n\to\infty} 2\frac{\ln(n)}{n} \right)} = 1,
			$$
			e a série converge para todo $z\in D(0,1)$.
			
			\item Temos que o raio $R$ de convergência é
			$$
			R = \frac{1}{\lim\limits_{n\to\infty} |\frac{1}{\ln(n)^n}|^{1/n} } = \lim\limits_{n\to\infty} |\ln(n)| = +\infty, 
			$$
			de modo que a série converge para todo $z\in\mathbb{C}$.
			\end{enumerate} 
		
		\item[25.] Supondo que $|z|<1$, podemos escrever
		\begin{align*}
		\frac{1}{1-z}f(z) &= \left(\sum_{n=0}^{\infty}z^n\right)\cdot\left(\sum_{n=0}^{\infty}a_nz^n\right) \\
		&= \sum_{n=0}^{\infty}(a_0+a_1+\cdots+a_n)z^n
		\end{align*}
		pela Proposição 2.7.
		
		\item[30.] 
		\begin{itemize}
			\item Temos que $0$ é o único ponto aderente de 
			$$
			\sqrt[n]{|a_n|} = \begin{cases}
			\displaystyle{\sqrt[n]{\frac{(n!)^2}{(2n)!}}}, n \text{ par} \\
			0, \text{ c.c.}
			\end{cases},
			$$
			de modo que 
			$$
			\limsup\sqrt[n]{|a_n|} = 0
			$$
			e
			$$
			R = \infty,
			$$
			e a série converge em todo $\mathbb{C}$.
			
			\item Temos que os pontos aderentes de 
			$$
			\sqrt[n]{|a_n|} = \begin{cases}
			\displaystyle{1+\frac{1}{n}}, n \text{ quadrado perfeito} \\
			0, \text{ c.c.}
			\end{cases},
			$$
			são $0$ e $1$, de modo que 
			$$
			\limsup\sqrt[n]{|a_n|} = 1
			$$
			e
			$$
			R = 1,
			$$
			e a série converge para todo $z\in D(0,1)$.
			
			\item Temos que os pontos aderentes de 
			$$
			\sqrt[n]{|a_n|} = \begin{cases}
			\displaystyle{\left(1+\frac{1}{n}\right)^n}, n \text{ quadrado perfeito} \\
			0, \text{ c.c.}
			\end{cases},
			$$
			são $0$ e $e$, de modo que 
			$$
			\limsup\sqrt[n]{|a_n|} = e
			$$
			e
			$$
			R = \frac{1}{e},
			$$
			e a série converge para todo $z\in D(0,1/e)$.
		\end{itemize}
	\end{enumerate}
\end{document}